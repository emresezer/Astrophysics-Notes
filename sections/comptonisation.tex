\section{Comptonisation}

Comptonisation, fotonların yüksek enerjili elektronlarla çoklu saçılmalar geçirmesi sonucunda ortalama enerjilerinin sistematik olarak artması ya da azalması sürecidir. Bu mekanizma özellikle X ışını ve gama ışını astrofiziğinde, akresyon disklerinin korona bölgelerinde ve sıcak plazmalarda tayfın sertleşmesine yol açan temel süreçlerden biridir.

\subsection{Temel Mekanizma}

Bir fotonun bir elektron tarafından saçılması Klein–Nishina veya Thomson rejiminde gerçekleşir. Enerji kazanımı, elektronların termal veya non-termal enerji dağılımına bağlıdır.

Tek saçılma için ortalama enerji değişimi:

\[
\Delta E \approx \frac{4kT_e - E}{m_e c^2} E
\]

\paragraph{Semboller:}
\begin{itemize}
    \item $E$: saçılmadan önceki foton enerjisi.
    \item $T_e$: elektron sıcaklığı.
    \item $k$: Boltzmann sabiti.
    \item $m_e$: elektron kütlesi.
\    \item $c$: ışık hızı.
\end{itemize}

$4kT_e > E$ ise fotonlar enerji kazanır ve süreç \textit{inverse Comptonisation} olarak adlandırılır.

\subsection{Compton $y$ Parametresi}

Comptonisation’ın gücünü belirleyen boyutsuz büyüklük:

\[
y = \frac{4kT_e}{m_e c^2} \max(\tau,\, \tau^2)
\]

\paragraph{Semboller:}

\begin{itemize}
    \item $y$: Compton güç parametresi.
    \item $T_e$: elektron sıcaklığı.
    \item $\tau$: elektronların optik derinliği.
\end{itemize}

$y \ll 1$ için Comptonisation zayıf olup fotonların enerjisi az değişir.  
$y \gtrsim 1$ için foton tayfı belirgin şekilde sertleşir.

\subsection{Komphton Tayfı ve Enerji Dağılımı}

Çoklu saçılmalar altında foton tayfı yaklaşık olarak bir güç yasası formuna yaklaşır:

\[
n(E) \propto E^{-\alpha}
\]

Burada foton indeksinin Compton parametresi ile ilişkisi:

\[
\alpha = -\frac{3}{2} + \sqrt{\frac{9}{4} + \frac{4}{y}}
\]

Bu sonuç, yüksek $y$ değerlerinin daha sert (daha düşük $\alpha$) bir tayfa karşılık geldiğini gösterir.

\subsection{Termal Comptonisation}

Elektronların Maxwell–Boltzmann dağılımına sahip olduğu bir plazmada fotonlar:

\[
E_{\mathrm{out}} \approx E_{\mathrm{in}}\, e^{y}
\]

şeklinde ortalama bir enerji artışı yaşar.

Termal Comptonisation, X ışını ikili sistemlerinde korona adı verilen sıcak ($T_e \sim 10^8$–$10^9\,\mathrm{K}$) bölgelerde gözlenen tipik güç yasası tayfını üretir.

\subsection{Non-termal Comptonisation}

Elektron popülasyonu bir güç yasası dağılımı ile tanımlandığında:

\[
N(E_e) \propto E_e^{-p}
\]

ortaya çıkan inverse Compton tayfı:

\[
L_\nu \propto \nu^{-(p-1)/2}
\]

şeklinde synchrotron mekanizmasına benzer bir güç yasası yapısı gösterir.

\subsection{Astrofiziksel Ortamlar}

\begin{itemize}
    \item Kara delik ve nötron yıldızı akresyon disklerinin sıcak korona bölgeleri.
    \item AGN (aktif galaksi çekirdeği) korona ve jet yapıları.
    \item Mikro-kuasar jetleri.
    \item Kozmik mikrodalga arka planı ile yüksek enerjili elektronların etkileşimi.
    \item Pulsar rüzgar bulutsuları ve şok bölgeleri.
\end{itemize}


\subsection{Ters Kompton Saçılması}

Ters Kompton saçılması (inverse Compton scattering), yüksek enerjili elektronların düşük enerjili fotonlara çarpması sonucunda fotonların enerji kazanarak daha yüksek frekanslara (genellikle X-ışını veya gamma bölgesine) çıkmasıdır. Astrofizikte yüksek enerjili foton üretiminin temel mekanizmalarından biridir.

Bu süreç özellikle şu ortamlarda kritik öneme sahiptir:

\begin{itemize}
    \item Aktif galaktik çekirdeklerin jetleri,
    \item Pulsar rüzgâr bulutsuları,
    \item Süpernova kalıntıları,
    \item Kozmik mikrodalga arka plan fotonlarının yüksek enerjili elektronlarla çarpışması (Sunyaev-Zel'dovich etkisi),
    \item X-ışını ikilileri (korona + disk etkileşimi).
\end{itemize}

\subsubsection*{Temel Fiziksel Mekanizma}

Bir elektronun fotona enerji aktarımı aşağıdaki temel bağıntıyla ifade edilir:

\[
E'_{\gamma} \approx \gamma^{2} E_{\gamma}
\]

\textbf{Açıklamalar:}
\begin{itemize}
    \item \( E_{\gamma} \): saçılmadan önceki foton enerjisi,
    \item \( E'_{\gamma} \): saçılma sonrası foton enerjisi,
    \item \( \gamma = \frac{1}{\sqrt{1-\beta^{2}}} \): elektronun Lorentz faktörü,
    \item \( \beta = v/c \): elektron hızının ışık hızına oranı.
\end{itemize}

Bu bağıntı ters Kompton sürecinin temel karakterini açıklar: fotonun enerjisi, elektronun Lorentz faktörünün karesi ile çarpılarak büyür. Dolayısıyla relativistik elektron popülasyonları bu mekanizmada belirleyici olur.

\subsubsection*{Klein–Nishina Rejimi ve Thomson Rejimi}

Ters Kompton saçılması iki rejimde incelenir:

\begin{itemize}
    \item \textbf{Thomson Rejimi:}
    \[
    E_{\gamma} \ll m_{e}c^{2}
    \]
    Foton elektronla etkileşirken enerjisi elektronun dinlenim enerjisinden çok küçüktür. Çapraz kesit klasik Thomson çapraz kesiti ile verilir:
    \[
    \sigma_{T} = \frac{8\pi}{3} r_{e}^{2}
    \]
    Burada \( r_{e} \) elektronun klasik yarıçapıdır.

    \item \textbf{Klein–Nishina Rejimi:}
    \[
    E_{\gamma} \sim m_{e}c^{2}
    \]
    Foton enerjisi elektronun dinlenim enerjisine yaklaşır, relativistik düzeltmeler gerekir. Klein–Nishina saçılma çapraz kesiti kullanılır.
\end{itemize}

\subsubsection*{Spektral Özellikler}

Elektron enerjileri bir güç kanunu dağılımı izliyorsa:
\[
N(E_{e}) \propto E_{e}^{-p}
\]

Ters Kompton yayınımının spektrumu da yaklaşık bir güç kanunu alır:
\[
L_{\nu} \propto \nu^{-(p-1)/2}
\]

Bu, synchrotron radyasyonu ile aynı elektron popülasyonunun üretmesi durumunda iki spektrum arasında doğrudan ilişki kurulabileceği anlamına gelir.

\subsubsection*{Astrofizikte Kullanım Alanları}

\begin{itemize}
    \item Jetlerdeki yüksek enerjili elektronların düşük enerjili disk fotonlarını X-ışınlarına yükseltmesi,
    \item Kozmik mikrodalga arka plan fotonlarının kümeler tarafından enerjilendirilmesi (SZ etkisi),
    \item Pulsar rüzgar bulutsularında elektron-popülasyon tanımlaması,
    \item Disk-korona modellerinde X-ışını spektrumunun yüksek enerjili kuyruğunu açıklama.
\end{itemize}

\subsubsection*{Enerji Kazancı}

Enerji aktarımının maksimum olduğu durum:
\[
E'_{\gamma, \max} \approx \frac{4}{3}\gamma^{2} E_{\gamma}
\]

Bu bağıntı, elektronun relativistik doğasının foton enerjisi üzerinde nasıl dramatik bir etki yarattığını gösterir.

```tex
%% Fiziksel Yorum – Not İçinde Kullanmak İçin: 
%% Bu konu özellikle X-ışını astrofiziği, AGN jetleri ve CMB-çevrimli foton artışları açısından doktora seviyesinde sıkça karşılaşılan temel mekanizmadır.
