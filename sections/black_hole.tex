\section{Kara Delikler}

Kara delikler, uzay-zamanda kütleçekimin o kadar yoğun olduğu bölgelerden oluşur ki ışık dahil hiçbir şey bu bölgeden kaçamaz. Kara deliğin merkezinde, kütleçekimin teorik olarak sonsuz olduğu \textbf{tekillik (singularity)} bulunur. Kara deliğin ``geri dönüşü olmayan sınırına'' \textbf{olay ufku} adı verilir.

Kara delikler, Genel Görelilik kuramının çözümlerine göre kütle, dönme (açısal momentum) ve elektrik yüküne göre sınıflandırılır. Üç temel model şunlardır: \textbf{Schwarzschild}, \textbf{Kerr} ve \textbf{Reissner–Nordström} kara delikleridir.

\subsection{Schwarzschild Kara Deliği (Dönmeyen ve Yüksüz)}

Schwarzschild kara deliği, dönmeyen ($J = 0$) ve yüksüz ($Q = 0$) ideal kara delik çözümüdür. Böyle bir kara deliğin olay ufku yarıçapı, yani \textbf{Schwarzschild yarıçapı}, şu şekilde verilir:

\begin{equation}
R_s = \frac{2GM}{c^2}
\end{equation}

Burada:
\[
R_s : \text{Schwarzschild yarıçapı}, \quad
G : \text{kütleçekim sabiti}, \quad
M : \text{kara delik kütlesi}, \quad
c : \text{ışık hızı}
\]

Bu yarıçap, olay ufkunun fiziksel büyüklüğünü belirler ve $M$ arttıkça doğrusal olarak artar.

\subsection{Kerr Kara Deliği (Dönen Kara Delikler)}

Gerçek astrofiziksel kara deliklerin çoğu açısal momentum taşır. Dönme etkili kara delikler \textbf{Kerr} çözümü ile tanımlanır. Dönme, tekilliği nokta halinden bir halka (ring singularity) şekline dönüştürür ve olay ufkunun yapısını değiştirir. Kerr kara deliğinde olay ufku yarıçapı:

\begin{equation}
r_{\pm} = \frac{GM}{c^2} \pm \sqrt{\left(\frac{GM}{c^2}\right)^2 - a^2}
\end{equation}

Burada:
\[
a = \frac{J}{Mc} : \text{dönme parametresi}
\]

\begin{itemize}
    \item $r_{+}$ dış olay ufku,
    \item $r_{-}$ iç olay ufku olarak adlandırılır.
\end{itemize}

Ayrıca Kerr kara deliklerinde olay ufkunun dışındaki bölgede \textbf{çekimsel sürüklenme (frame dragging)} oluşur. Bu bölgeye \textbf{ergosfer} denir. Ergoda hiçbir cisim uzay-zamanın dönme yönüne karşı hareket edemez.

\subsection{Reissner–Nordström Kara Deliği (Yüklü Kara Delikler)}

Kara delik elektrik yükü taşıyorsa ($Q \neq 0$) dönme yoksa ($J = 0$), bu yapı \textbf{Reissner–Nordström} kara deliği olarak tanımlanır. Bu durumda olay ufku iki adet olur:

\begin{equation}
r_{\pm} = \frac{GM}{c^2} \pm \sqrt{\left(\frac{GM}{c^2}\right)^2 - \frac{G Q^2}{4 \pi \varepsilon_0 c^4}}
\end{equation}

\begin{itemize}
    \item $r_{+}$ dış olay ufku,
    \item $r_{-}$ iç olay ufku.
\end{itemize}

Eğer:
\[
\left(\frac{GM}{c^2}\right)^2 < \frac{G Q^2}{4 \pi \varepsilon_0 c^4}
\]
olursa olay ufku kalmaz ve \textbf{çıplak tekillik} ortaya çıkar. Bu durum \textbf{kozmos sansür hipotezi} gereği fiziksel olarak gerçekleşmemesi gereken bir durum olarak değerlendirilir.
