\section{Bremsstrahlung (Frenleme Işıması)}

Bremsstrahlung, yüksek enerjili bir yüklü parçacığın (genellikle elektron) bir elektrik alan içinde ivmelenmesi veya yavaşlaması sonucunda ortaya çıkan sürekli tayflı elektromanyetik ışınımdır. Astrofizikte sıcak plazmalarda, yıldız atmosferlerinde, galaksi kümelerinde ve X ışını kaynaklarında temel yayınım mekanizmalarından biridir.

\subsubsection{Fiziksel Mekanizma}

Yüklü bir parçacığın ivmeli hareketi, Maxwell denklemlerine göre elektromanyetik dalga yaymasına yol açar. Elektron bir iyonun ($Z$ yüklü çekirdek) elektrik alanından geçerken yön değiştirir ve enerji kaybederek foton üretir:

\[
e^- + Z \rightarrow e^- + Z + h\nu
\]

Bu süreç, tek bir saçılmada sürekli enerji dağılımına sahip foton üretir. Yüksek sıcaklıklarda (örn. $T \gtrsim 10^6\,\mathrm{K}$) plazma, güçlü X ışını bremsstrahlung yayınımı üretir.

\subsubsection{Spektral Enerji Dağılımı}

Termal bremsstrahlung yayınımının hacim başına spektral güç yoğunluğu:

\[
\varepsilon_\nu = 6.8\times 10^{-38}\, Z^2\, n_e\, n_i\,
T^{-1/2}\, e^{-h\nu/kT}\, \bar{g}_{\mathrm{ff}}
\]

\paragraph{Semboller:}
\begin{itemize}
    \item $\varepsilon_\nu$: birim hacim başına birim frekanstaki güç (erg\,s$^{-1}$\,cm$^{-3}$\,Hz$^{-1}$).
    \item $Z$: iyonun yük sayısı.
    \item $n_e$: elektron yoğunluğu (cm$^{-3}$).
    \item $n_i$: iyon yoğunluğu (cm$^{-3}$).
    \item $T$: plazma sıcaklığı (K).
    \item $h$: Planck sabiti.
    \item $\nu$: foton frekansı (Hz).
    \item $k$: Boltzmann sabiti.
    \item $\bar{g}_{\mathrm{ff}}$: Gaunt faktörü (kuantum düzeltmesi, $\sim 1$).
\end{itemize}

\subsubsection{Toplam Yayınım Gücü}

Termal bremsstrahlung için tüm frekanslara entegre edilmiş toplam hacimsel güç:

\[
\varepsilon = 1.4\times 10^{-27}\, Z^2\, n_e\, n_i\, T^{1/2}
\]

Bu ifade, sıcaklığın artmasıyla yayınım gücünün yükseldiğini ve yoğunluğun iki parçacık türünün çarpımıyla orantılı olduğunu gösterir.

\subsubsection{Astrofiziksel Ortamlar}

\begin{itemize}
    \item $T \sim 10^6$–$10^8\,\mathrm{K}$ sıcaklığındaki galaksi kümelerinin X ışını yayınımı.
    \item Süpernova artıklarının sıcak plazmaları.
    \item Akresyon disklerinin iç kısımlarındaki iyonize gaz.
    \item Yıldız atmosferlerindeki yüksek sıcaklıklı bölgeler.
\end{itemize}

\subsubsection{Temel Özellikler}

\begin{itemize}
    \item Sürekli tayf üretir.
    \item Yüksek sıcaklıkta baskın X ışını mekanizmasıdır.
    \item Güç yoğunluğu elektron–iyon çarpışmalarının sıklığına bağlıdır.
    \item Termal bremsstrahlung spektrumu $\exp(-h\nu/kT)$ ile belirlenen bir kesim frekansına sahiptir.
\end{itemize}

\begin{figure}[ht!]
    \centering
    \resizebox{0.5\textwidth}{!}{\includegraphics{figures/bremsstrahlung.png}}
    \caption{Bremsstrahlung, atom çekirdeğinin elektrik alanında saptırılan yüksek enerjili bir elektron tarafından üretilir. Makine tarafından okunabilen yazar sağlanmadı. Journey234 varsayılmıştır (telif hakkı taleplerine dayanarak). - Kendi çalışması: Martin, Dylan (2005). X-Işını Algılama. Arizona Üniversitesi Optik Bilimler Merkezi. 2006-09-09 tarihinde orjinalinden arşivlendi. Erişim tarihi: 2008-12-05. Flynn, Chris (2001). 6.6.2 Serbest-Serbest veya Bremsstrahlung Saçılımı. Tuorla Gözlemevi. Erişim tarihi: 2008-12-05. Erzeugung von Röntgenbremsstrahlung durch Abbremsung, Coulombfeld'deki Elektronlar ve Atom Çekirdekleri (Darstellung Şeması)}
    \label{fig:HRD}
\end{figure}