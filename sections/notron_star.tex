\section{Nötron Yıldızları}

Nötron yıldızları, büyük kütleli yıldızların (yaklaşık $8M_{\odot} \lesssim M \lesssim 25M_{\odot}$) çekirdek çökmesi sonucu oluşan, son derece yoğun ve kompakt yapılardır. Süpernova patlaması ile dış katmanlar uzaya saçılırken çekirdek, protonların ve elektronların birleşerek nötron oluşturmasıyla nötronca zengin ve yoğun bir yapıya dönüşür. Nötron yıldızlarının tipik özellikleri:

\[
R \approx 10 \ \text{km}, \qquad M \approx 1.4-2.5 M_{\odot}, \qquad \rho \approx 10^{14} \ \text{g/cm}^3
\]

Bu yoğunluk, bir çay kaşığı nötron yıldızı maddesinin milyonlarca ton kütleye denk gelmesi anlamına gelir. Nötron yıldızlarında denge, kütleçekim kuvvetine karşı \textbf{nötron dejenere basıncı} ile sağlanır.

\subsection{Pulsarlar}

Pulsar, yüksek manyetik alan ($10^{8} - 10^{15}$ Gauss) ve çok hızlı dönme periyoduna sahip nötron yıldızlarıdır. Manyetik eksen ile dönme ekseni aynı doğrultuda olmadığından, yıldız döndükçe manyetik kutuplardan yayılan elektromanyetik ışınım, uzaktaki bir gözlemci için periyodik bir \textbf{radyo darbesi (pulse)} şeklinde algılanır.

\begin{itemize}
    \item Pulsarların dönüş periyodu milisaniyeler ile saniyeler arasında değişir.
    \item Zaman içindeki dönme yavaşlaması, manyetik dipol ışıması ile açıklanır.
\end{itemize}

Pulsarın dönme periyodu:
\[
P = \text{Dönme Periyodu}, \qquad \dot{P} = \text{Periyottaki yavaşlama}
\]

Pulsarın yaş tahmini (spin-down age):
\[
\tau \approx \frac{P}{2 \dot{P}}
\]

\subsection{Akresyon Diski}

Eğer nötron yıldızı bir \textbf{ikili yıldız sisteminde} yer alıyorsa ve yoldaş yıldızdan madde çekiyorsa, üzerine düşen madde açısal momentum nedeniyle doğrudan yüzeye çökemez. Bunun yerine nötron yıldızının etrafında bir \textbf{akresyon diski} oluşur.

Bu diskte:
\begin{itemize}
    \item Madde iç sürtünme nedeniyle ısınır,
    \item Termal radyasyon ve X-ışınları yayılır,
    \item Manyetik alan çizgileri, maddenin kutuplara yönelmesine neden olabilir.
\end{itemize}

Akresyon sonucu oluşan enerji ışıması:
\[
L_{\text{akresyon}} \approx \frac{G M \dot{M}}{R}
\]

Burada:
\[
M : \text{nötron yıldızı kütlesi}, \qquad R : \text{nötron yıldızı yarıçapı}, \qquad \dot{M} : \text{akresyon hızı}
\]

Akresyon süreçleri, nötron yıldızlarını \textbf{X-ışını ikilileri (X-ray binaries)} olarak gözle görünür hale getirir. Eğer akresyon nedeniyle açısal momentum artarsa, nötron yıldızı hızlanarak \textbf{milisaniye pulsarına} dönüşebilir.