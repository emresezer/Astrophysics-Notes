 \section{Planck Işıma Yasası}

Termal dengedeki bir siyah cisim tarafından yayılan elektromanyetik ışınımın spektral enerji yoğunluğu Planck yasası ile verilir. Bu yasa, klasik fiziğin ultraviolet felaketini çözerek kuantum kavramını ortaya çıkaran temel sonuçlardan biridir.

\[
B_\nu(T) = \frac{2h\nu^3}{c^2}\,\frac{1}{e^{h\nu/kT}-1}
\]

\paragraph{Semboller:}
\begin{itemize}
    \item $B_\nu(T)$: Birim frekansta birim yüzeyden yayılan özgül şiddet (erg\,s$^{-1}$\,cm$^{-2}$\,Hz$^{-1}$\,sr$^{-1}$).
    \item $\nu$: Işımanın frekansı (Hz).
    \item $T$: Sıcaklık (K).
    \item $h$: Planck sabiti.
    \item $k$: Boltzmann sabiti.
    \item $c$: Işığın hızı.
\end{itemize}

Planck yasasının temel özelliği, yüksek frekansta üstel bir azalma göstermesidir:
\[
B_\nu \propto \nu^3 e^{-h\nu/kT}
\]


\section{Wien Yer Değiştirme Yasası}

Planck fonksiyonunun maksimum yaptığı dalgaboyu sıcaklıkla ters orantılıdır. Bu bağıntı Wien yasasıdır:

\[
\lambda_{\max} T = 2.898\times 10^{-3}\, \mathrm{m\,K}
\]

\paragraph{Açıklama:}
\begin{itemize}
    \item $\lambda_{\max}$: Spektrumun maksimum şiddet verdiği dalgaboyu.
    \item $T$: Siyah cismin sıcaklığı.
\end{itemize}

Bu yasa sayesinde yıldızların sıcaklığı yalnızca tepe dalgaboyu ölçülerek belirlenebilir. Örneğin Güneş için $\lambda_{\max} \approx 500\,\mathrm{nm}$, buradan $T \approx 5800\,\mathrm{K}$ elde edilir.


\section{Stefan–Boltzmann Yasası}

Siyah cismin tüm dalgaboyları boyunca toplam yayılan gücü sıcaklığın dördüncü kuvvetiyle orantılıdır:

\[
F = \sigma T^4
\]

\paragraph{Semboller:}
\begin{itemize}
    \item $F$: Birim yüzeyden yayılan toplam enerji akısı (erg\,s$^{-1}$\,cm$^{-2}$).
    \item $\sigma$: Stefan–Boltzmann sabiti.
\end{itemize}

Stefan–Boltzmann sabitinin teorik ifadesi:

\[
\sigma = \frac{2\pi^5 k^4}{15 h^3 c^2}
\]

Bu, kuantum mekaniği ve istatistiksel fiziğin birleştiği en önemli sonuçlardan biridir.


\section{Larmor Frekansı}

Bir manyetik alan içinde hareket eden yüklü parçacıklar dairesel bir yörüngede dönerek karakteristik bir açısal hızla titreşir. Bu frekans Larmor frekansı olarak adlandırılır:

\[
\omega_L = \frac{qB}{m}
\]

\paragraph{Semboller:}
\begin{itemize}
    \item $\omega_L$: Larmor açısal frekansı (rad\,s$^{-1}$).
    \item $q$: Parçacığın elektrik yükü.
    \item $B$: Manyetik alan şiddeti.
    \item $m$: Parçacığın kütlesi.
\end{itemize}

Bu frekans, siklotron ve senkrotron ışınımlarının temelini oluşturur. Enerjisi düşük (relativistik olmayan) elektronlar siklotron yayınımı, yüksek enerjili olanlar ise senkrotron yayınımı üretir.


\section{Beaming (Işın Daralması) Etkisi}

Relativistik hızlarda hareket eden bir kaynak tarafından yayılan ışınım Lorentz dönüşümlerinden dolayı gözlemci doğrultusunda dar bir koni içine sıkışır. Bu olaya \textit{relativistik beaming} denir.

\[
\theta \sim \frac{1}{\gamma}
\]

\paragraph{Semboller:}
\begin{itemize}
    \item $\theta$: Yayınımın yoğunlaştığı koni açısı.
    \item $\gamma = (1-v^2/c^2)^{-1/2}$: Lorentz faktörü.
\end{itemize}

Beaming sonucu:
\begin{itemize}
    \item Işınım gözlemci doğrultusunda güçlenir.
    \item Jet yapılar (AGN'ler, GRB'ler) daha parlak ve değişken görünür.
    \item Doppler yükseltmesi nedeniyle gözlenen parlaklık artar.
\end{itemize}


\section{Planck–Einstein İlişkisi}

Bir fotonun enerjisi frekansı ile doğru orantılıdır. Kuantum mekaniğinin en temel ilişkilerinden biri olan Planck–Einstein denklemi:

\[
E = h\nu
\]

Ayrıca dalga-boyu üzerinden ifade edilirse:

\[
E = \frac{hc}{\lambda}
\]

\paragraph{Semboller:}
\begin{itemize}
    \item $E$: Foton enerjisi.
    \item $\nu$: Frekans.
    \item $\lambda$: Dalgaboyu.
    \item $h$: Planck sabiti.
    \item $c$: Işığın hızı.
\end{itemize}

Bu ilişki, fotoelektrik olaydan Compton saçılmasına kadar tüm kuantum ışınım süreçlerinin temelidir.
