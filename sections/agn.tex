\section{Aktif Galaktik Çekirdek (AGN)}

Aktif Galaktik Çekirdek (AGN), galaksilerin merkezinde bulunan, büyük kütleli bir karadeliğin etrafındaki akresyon diski tarafından üretilen aşırı parlak elektromanyetik ışınım kaynaklarıdır. Yaydıkları enerji, tipik bir galaksinin tüm yıldızlarından daha parlak olabilir. AGN'ler çok geniş bir spektral bantta (radyo–X ışını–gamma) ışınım yapar ve evrendeki en enerjik astrofiziksel yapılardandır.

\subsection{Temel Fiziksel Mekanizma}

AGN’nin güç kaynağı, kütlesi genellikle
\[
10^6 - 10^{10}\, M_\odot
\]
aralığında olan süper kütleli karadeliktir. Karadeliğe düşen gaz, akresyon sırasında açısal momentum kaybederek ısınır ve güçlü bir radyasyon üretir.

Akresyon diskinin parlaklığı temel olarak şu bağıntıyla sınırlanır:

\[
L_{\mathrm{Edd}} = 1.3 \times 10^{38} \left( \frac{M}{M_\odot} \right) \, \mathrm{erg\,s^{-1}}
\]

\paragraph{Semboller:}
\begin{itemize}
    \item $L_{\mathrm{Edd}}$: Eddington parlaklığı (radyasyon basıncı–yerçekimi dengesi sınırı).
    \item $M$: Merkezdeki karadeliğin kütlesi.
    \item $M_\odot$: Güneş kütlesi.
\end{itemize}

Akresyon diskinin ışıma gücü ise:
\[
L = \eta\, \dot{M} c^2
\]
Burada $\eta \sim 0.1$ tipik verim katsayısıdır.

\subsection{Gözlemsel Özellikler}

AGN'ler çok geniş bir spektral enerji dağılımına sahiptir. Yayınım şu bileşenleri içerir:

\begin{itemize}
    \item Termal UV–optik ışınım (akresyon diski).
    \item X-ışını yayınımı (korona).
    \item Radyo yayınımı (jetler ve senkrotron süreç).
    \item İnfrared yayınım (toz torusu).
\end{itemize}

\subsection{Relativistik Jetler}

Bazı AGN'ler, ışık hızına çok yakın hızlarla dışarı doğru fırlayan relativistik jetler üretir. Jetlerde beaming etkisi gözlenir:

\[
\theta \approx \frac{1}{\gamma}
\]

\subsection{AGN Türleri}

\begin{itemize}
    \item \textbf{Seyfert Galaksileri:} Çekirdekleri parlak, spiral gökadalarda görülür.
    \item \textbf{Kuazarlar:} Evrendeki en parlak AGN türüdür; çok uzaklarda bulunur.
    \item \textbf{Blazarlar:} Jet doğrultusu Dünya’ya dönük olan AGN; beaming nedeniyle aşırı parlak görünür.
    \item \textbf{Radyo Galaksileri:} Dev radyo loblarına sahip AGN'lerdir.
\end{itemize}

\subsection{Astrofiziksel Önemi}

\begin{itemize}
    \item Evrende baryonik maddenin büyük kısmının termal ve dinamik evrimini etkiler.
    \item Geri besleme (feedback) mekanizmasıyla galaksi oluşumunu düzenler.
    \item Kozmik mesafe belirteci olarak kullanılır (özellikle kuazarlar).
    \item Yüksek enerji astrofiziğin laboratuvarı niteliğindedir.
\end{itemize}

