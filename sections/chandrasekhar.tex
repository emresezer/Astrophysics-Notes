\section{Chandrasekhar Limiti}

Beyaz cüceler, çökmeye çalışan kütleçekim kuvvetine karşı elektronların oluşturduğu \textit{Dejenere Elektron Basıncı} ile dengede kalırlar. Ancak elektronlar ışık hızına yaklaşan (relativistik) bir rejime sıkıştıklarında bu basınç yerçekimini dengeleyemez hale gelir. Bu kritik kütleye \textbf{Chandrasekhar Limiti} denir ve bu değerin üzerinde beyaz cüce çöker. Yani kararlı bir Beyaz Cüce'nin sahip olabileceği en yüksek kütle sınırıdır.

\[
P_{\text{deg}} \propto \left( \frac{\rho}{\mu_e} \right)^{4/3}, \qquad
P_{\text{grav}} \propto \frac{G M^2}{R^4}
\]

Denge koşulu:

\[
P_{\text{deg}} = P_{\text{grav}}
\]

Bu eşitliğin çözümü beyaz cücenin yarıçapının kütleyle ters orantılı olduğunu verir:

\[
R \propto M^{-1/3}
\]

Yani kütle arttıkça beyaz cüce küçülür ve yoğunlaşır. Kritik kütle ise:

\[
M_{\text{Ch}} = \frac{5.83}{\mu_e^2} M_\odot
\]

Karbon-Oksijen beyaz cüceleri için \(\mu_e \approx 2\):

\[
M_{\text{Ch}} \approx \frac{5.83}{(2)^2} M_\odot \approx 1.44 M_\odot
\]

\noindent
\textbf{Sonuç:} Bir beyaz cücenin kütlesi \(\sim 1.44 M_\odot\) değerini aşarsa elektron yozlaşma basıncı çökmeyi durduramaz. Yıldız nötron yıldızına çöker veya Tip Ia süpernova gerçekleştirir.
