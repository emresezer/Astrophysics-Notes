\subsection{Roche Potansiyeli ve Roche Lobu}

Çift yıldız sistemlerinde, iki yıldız ortak bir kütle merkezi etrafında döner. Bu durumda sistemin dinamiğini belirleyen kuvvetler yalnızca kütleçekimi değildir; aynı zamanda dönme hareketinden kaynaklanan merkezkaç etkisi de vardır. İki yıldız arasındaki etkileşim, \textbf{Roche potansiyeli} olarak adlandırılan etkin kütleçekim potansiyeli ile tanımlanır.

Roche potansiyeli, iki yıldızın kütleçekim alanını ve dönme çerçevesindeki merkezkaç kuvvetini birlikte ifade eder. İki kütle $M_1$ ve $M_2$ için dönme çerçevesindeki etkili potansiyel:

\[
\Phi(\mathbf{r}) = -\frac{G M_1}{|\mathbf{r}-\mathbf{r_1}|} - \frac{G M_2}{|\mathbf{r}-\mathbf{r_2}|} - \frac{1}{2}\omega^2 |\mathbf{r}-\mathbf{r_c}|^2
\]

Burada:
\[
\omega = \sqrt{\frac{G(M_1 + M_2)}{a^3}}
\]
ikili sistemin açısal hızıdır, $a$ yıldızlar arası uzaklıktır ve $\mathbf{r_c}$ ortak kütle merkezidir.

Bu potansiyelin seviyelendirilmiş yüzeyleri (eş potansiyel yüzeyleri), sistemdeki madde akışının sınırlarını belirler. \textbf{Roche lobu}, her bir yıldızın kütleçekimsel olarak hâkim olduğu hacmi belirleyen kapalı eş potansiyel yüzeyidir.

\[
\textbf{Roche lobu = yıldızın kendine ait tutabileceği maksimum hacim.}
\]

Roche lobunun yaklaşık yarıçapı Eggleton formülü ile ifade edilir:

\[
\frac{R_{L1}}{a} = \frac{0.49\, q^{2/3}}{0.6\, q^{2/3} + \ln(1+q^{1/3})}
\]

Burada:
- $q = \dfrac{M_1}{M_2}$ kütle oranı,
- $R_{L1}$ birinci yıldızın Roche lobu yarıçapı,
- $a$ yıldızlar arası uzaklıktır.

\subsubsection*{L1 Noktası ve Madde Akışı}

İki Roche lobu arasında, \textbf{Lagrange Noktası L1} adı verilen kararlı bir denge noktası bulunur. Eğer yıldızlardan biri evrimsel süreçte genişleyip \textbf{Roche lobu sınırını aşarsa}, dış katmanlar bu dar boğaz noktasından diğer yıldıza akar:

\[
\textbf{Roche Lobu Taşması → Kütle Transferi}
\]

Bu süreç:

- Yakın çift yıldız evrimini belirleyen ana mekanizmadır.
- Sistem momentumunu, yörüngesel dönme oranlarını ve yıldızların parlaklıklarını değiştirir.
- Akış genellikle L1 etrafında bir \textbf{akresyon diski} oluşturarak gerçekleşir (beyaz cüce, nötron yıldızı veya kara delik varsa).

\subsubsection*{Özet (Büyük Resim)}

\begin{itemize}
\item Roche potansiyeli, kütleçekim + merkezkaç etkisini birlikte tanımlar.
\item Her yıldızın maddeyi elinde tutabildiği hacim \textbf{Roche lobudur}.
\item Yıldız genişleyip Roche lobunu aşarsa \textbf{kütle transferi başlar}.
\item Kütle transferi, sistemin tüm evrimini kökten değiştirir.
\end{itemize}

\subsection{Kütle Transferi Türleri}

Yakın çift yıldızlarda kütle transferi, Roche lobu taşması veya yıldız rüzgarları yoluyla gerçekleşir. Kütle transferi sürecinin doğası, bileşen yıldızların evrimsel aşamalarına ve kütle oranına bağlıdır. Yaygın kütle transferi türleri aşağıda verilmiştir:

\begin{itemize}
    \item \textbf{Roche Lobu Taşması (Roche Lobe Overflow, RLOF):} 
    Bir yıldızın yarıçapı Roche lobu sınırını aştığında, madde iç Lagrange noktası $L_1$ üzerinden diğer bileşene akar. Bu süreç yönlendirilmiş ve yüksek akılı bir aktarım mekanizmasıdır.

    \item \textbf{Yıldız Rüzgarı ile Kütle Aktarımı:}
    Geniş çift yıldız sistemlerinde büyük yarıçaplı yıldızlar güçlü yıldız rüzgarları yayar. Bu rüzgarın bir kısmı ikinci bileşen tarafından yakalanarak akresyon sağlar. Akresyon verimi düşük fakat kararlı olabilir.

    \item \textbf{Ortak Zarf Evrimi (Common Envelope, CE):}
    Kütle transferi dengesiz hâle geldiğinde alıcı yıldız gelen maddeyi tutamaz ve iki yıldız ortak bir gaz zarfı içerisine gömülür. Bu süreçte yörünge yarıçapı dramatik biçimde küçülür ve enerji açığa çıkar. CE fazı kompakt çift yıldızların ana oluşum mekanizmalarından biridir.
\end{itemize}

\subsection{Kütle Transferi Kararlılığı}

Kütle transferinin kararlı olup olmaması, yıldızın kütle kaybına tepkisine ve sistemdeki kütle oranına bağlıdır. Kararlılık analizi, donor yıldızın yarıçap tepkisi ile Roche lobunun yarıçap değişimini karşılaştırarak yapılır.

\begin{itemize}
    \item Eğer donor yıldızın yarıçap değişimi Roche lobunun genişleme hızından \textbf{daha yavaş} ise:
    \[
    \frac{d\ln R_{\mathrm{donor}}}{d\ln M_{\mathrm{donor}}} > \frac{d\ln R_{L}}{d\ln M_{\mathrm{donor}}}
    \]
    \textbf{Kütle transferi kararlıdır}. Sistem uzun süreli ve düzenli akresyon gösterir.

    \item Eğer donor yıldız kütle kaybederken \textbf{genişliyor} ve Roche lobu onu dar bir şekilde sınırlandırıyorsa:
    \[
    \frac{d\ln R_{\mathrm{donor}}}{d\ln M_{\mathrm{donor}}} < \frac{d\ln R_{L}}{d\ln M_{\mathrm{donor}}}
    \]
    \textbf{Kütle transferi kararsızdır}. Bu durumda sistem hızlı kütle aktarım fazına girer ve çoğu zaman \textbf{ortak zarf evrimi (CE)} ile sonuçlanır.
\end{itemize}

\subsubsection{Kütle Oranı Kriteri}

Kütle transferi kararlılığı çoğunlukla kütle oranı ile karakterize edilir:
\[
q = \frac{M_{\mathrm{donor}}}{M_{\mathrm{accretor}}}
\]

\begin{itemize}
    \item $q < q_{\mathrm{kritik}}$ ise kütle transferi \textbf{kararlıdır}.
    \item $q > q_{\mathrm{kritik}}$ ise kütle transferi \textbf{kararsızdır} ve CE fazına yol açabilir.
\end{itemize}

Kritik kütle oranı yıldızın iç yapısına göre değişir:
\[
q_{\mathrm{kritik}} \approx
\begin{cases}
1.5 - 3.0, & \text{radyatif zarf yapılı donorlar} \\
0.6 - 1.0, & \text{konvektif zarf yapılı donorlar}
\end{cases}
\]
