\section{Çift Yıldız Sistemleri}

Çift yıldızlar, kütleçekimsel olarak birbirine bağlı iki yıldızdan oluşan sistemlerdir. Bu yıldızlar ortak bir kütle merkezi etrafında dönerek dinamik ve evrimsel açıdan birbirlerini etkilerler. Evren'de yıldızların önemli bir kısmı tek değil, çift veya çoklu sistemler halinde bulunur. Çift yıldızlar hem yıldız evrimi hem de kütle kaybı / kazanımı süreçlerinin anlaşılmasında kritik rol oynar.

Çift yıldızlar, bileşen yıldızların birbirlerine olan uzaklıklarına ve aralarındaki kütle aktarımına göre iki ana kategoride incelenir:

\subsection{Geniş Çift Yıldızlar (Wide Binaries)}

Geniş çift yıldızlarda yıldızlar arasındaki uzaklık büyük olduğu için Roche lobları dolmaz ve \textbf{kütle aktarımı gerçekleşmez}. Her iki yıldız evrimini neredeyse tek yıldızmış gibi bağımsız sürdürür. Bu sistemlerde bileşenler birbirlerini yalnızca kütleçekimsel olarak etkiler.

\subsection{Yakın Çift Yıldızlar (Close Binaries)}

Yakın çift yıldızlarda bileşenler birbirine yeterince yakın olduğundan \textbf{Roche lobu taşması} meydana gelebilir. Bu durumda daha büyük ve gelişmiş olan yıldız, \textbf{kütle transferi} ile maddeyi diğer bileşene aktarır. Sistem içerisindeki evrim, tek yıldız evriminden önemli ölçüde farklılaşır.

\subsection{Geniş ve Yakın Çift Yıldızların Evrimsel Sonuçları}

Aşağıdaki tablo, çift yıldız türlerini ve bileşenlerin evrimsel sonuçlarını özetler:

\begin{table}[h!]
\centering
\begin{tabular}{|p{0.22\textwidth}|p{0.22\textwidth}|p{0.50\textwidth}|}
\hline
\textbf{1. Bileşen} & \textbf{2. Bileşen} & \textbf{Sistem Sonucu} \\
\hline
Kırmızı Dev & Ana Kol Yıldızı & Geniş çift sistem; kütle aktarımı yok. \\
\hline
Kırmızı Dev & Ana Kol Yıldızı & Yakın çift sistem; Roche taşması ile kütle transferi. \\
\hline
Beyaz Cüce & Ana Kol Yıldızı & Akresyon → Nova; $M_{\mathrm{Ch}}$ aşılırsa Tip Ia Süpernova. \\
\hline
Nötron Yıldızı & Ana Kol Yıldızı & X-ışını ikilisi; akresyon diski yüksek enerjili yayım üretir. \\
\hline
Nötron Yıldızı & Nötron Yıldızı & Birleşme → Gravitasyonel dalga + Kilonova. \\
\hline
Kara Delik & Ana Kol Yıldızı & Mikrokuasar; jetler ve güçlü akresyon diski. \\
\hline
Kara Delik & Kara Delik & Gravitasyonel dalga ile birleşme olayı (GW sinyali). \\
\hline
\end{tabular}
\caption{Çift yıldız bileşenlerine göre evrimsel sonuçlar.}
\end{table}
