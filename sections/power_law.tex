\section{Power-Law (Güç Kanunu)}

Güç kanunu, bir fiziksel niceliğin başka bir nicelikle basit bir üstel oranda ölçeklendiğini ifade eder. Astrofizikte enerji tayfları, parçacık dağılımları, manyetik alan spektrumları ve jet yayınımları çoğunlukla power-law biçimde tanımlanır.


Bir büyüklüğün power-law ile ölçeklenmesi şu şekilde ifade edilir:

\[
F(x) = A\, x^{-\alpha}
\]

\paragraph{Semboller:}
\begin{itemize}
    \item $F(x)$: İncelenen fiziksel büyüklük (ör. akı, yoğunluk, sayım oranı).
    \item $A$: Normalizasyon sabiti.
    \item $\alpha$: Spektral indeks (eğim).
    \item $x$: Bağımsız değişken (enerji, frekans, momentum vb.).
\end{itemize}

Power-law fonksiyonları logaritmik eksenlerde düz bir doğru verir:

\[
\log F = \log A - \alpha \log x
\]

Bu özellik, gözlemsel astrofizikte spektral eğimleri belirlemek için kullanılır.

\subsection{Astrofizikte Kullanım Alanları}

\begin{itemize}
    \item \textbf{Senkrotron tayfı:} Relativistik elektron dağılımları power-law biçimindedir.
    \[
    N(E) \propto E^{-p}
    \]
    \item \textbf{Kuazar ve AGN X-ışını tayfı:}
    \[
    F_E \propto E^{-\Gamma}
    \]
    Burada $\Gamma$ foton indeksi olup tipik değerleri $1.5$–$2.0$ aralığındadır.
    \item \textbf{Karadelik ve nötron yıldızı jetleri.}
    \item \textbf{Kozmik ışın enerji dağılımları.}
\end{itemize}

\subsection{Power-Law Dağılımlarının Özellikleri}

\begin{itemize}
    \item Yüksek enerjelerde yavaş zayıflayan kuyruk üretir (heavy tail).
    \item Belirli momentler (ortalama, varyans) bazı $\alpha$ değerlerinde diverge olabilir.
    \item Astrofizikte çoğunlukla non-termal süreçlerin göstergesidir.
\end{itemize}

\subsection{Senkrotron Spektral İndeksi Bağıntısı}

Relativistik elektronların enerji dağılımı power-law ise:

\[
N(E) \propto E^{-p}
\]

Yayınım spektrumunun eğimi:

\[
F_\nu \propto \nu^{-(p-1)/2}
\]

\paragraph{Bağıntının Önemi:}
Bu ilişki gözlenen radyo veya X-ışını eğimini kullanarak elektronların enerji dağılım indeksini belirleme imkânı sağlar.
