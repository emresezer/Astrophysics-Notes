\section{Spektral Sınıflandırma}

Yıldızların spektral sınıflandırması, yüzey sıcaklığına ve tayf çizgilerine göre yapılır. Sınıflar en sıcaktan soğuğa doğru aşağıdaki gibidir:

\begin{itemize}
    \item \textbf{O Sınıfı}: 
    Çok sıcak (30\,000--50\,000\,K), mavi renklidir.
    Tayflarında güçlü \textit{He\,II} ve zayıf \textit{H} çizgileri görülür.

    \item \textbf{B Sınıfı}: 
    Sıcaklığı 10\,000--30\,000\,K aralığındadır, mavi-beyaz görünür.
    \textit{He\,I} çizgileri belirgindir.

    \item \textbf{A Sınıfı}: 
    7\,500--10\,000\,K aralığında, beyaz renklidir.
    En güçlü \textit{Balmer} (H) çizgileri bu sınıfta görülür.

    \item \textbf{F Sınıfı}: 
    6\,000--7\,500\,K arası, sarı-beyaz görünür.
    \textit{Ca\,II H-K} çizgileri belirginleşir.

    \item \textbf{G Sınıfı}: 
    5\,200--6\,000\,K, sarı renkli yıldızlar (Güneş bu sınıftadır).
    Metal soğurma çizgileri ve \textit{Ca\,II H-K} çizgileri güçlüdür.

    \item \textbf{K Sınıfı}: 
    3\,700--5\,200\,K, turuncu renkte.
    Metal çizgileri ve moleküler bantlar belirginleşmeye başlar.

    \item \textbf{M Sınıfı}: 
    2\,400--3\,700\,K, kırmızı renktedir.
    \textit{TiO} (titanyum oksit) bantları karakteristiktir.
\end{itemize}

Sıcaklık sıralaması: 

\[
\textbf{O > B > A > F > G > K > M}
\]

Renk ve sıcaklık ilişkisi: sıcak yıldızlar mavi, soğuk yıldızlar kırmızı görünür. Her alt sınıf kendi içerisinde 0-9 sınıfa ayrılır.

\subsection{Parlaklık (Luminosity) Sınıflandırması}

Yıldızlar yalnızca sıcaklıklarına (spektral sınıf) göre değil,
aynı zamanda parlaklıkları ve yarıçapları dikkate alınarak
\textbf{Roma rakamları} ile sınıflandırılır:

\begin{itemize}
    \item \textbf{I — Süperdev (Supergiant)} \\
    Çok büyük yarıçaplı ve çok parlak yıldızlar. 
    Genellikle evrimlerinin ileri aşamalarındadırlar. 
    Alt sınıfları: Ia (çok parlak), Ib (daha sönük süperdev).

    \item \textbf{II — Parlak Dev (Bright Giant)} \\
    Dev yıldızlardan daha parlak, süperdevlerden daha küçüktür.
    Evrimsel olarak kırmızı dev aşamasına yakın yıldızları içerir.

    \item \textbf{III — Dev (Giant)} \\
    Hidrojen kabuk yanmasının sürdüğü, genişlemiş yarıçaplı yıldızlar.
    Kırmızı devlerin çoğu bu sınıfa girer.

    \item \textbf{IV — Alt Dev (Subgiant)} \\
    Anakol aşamasından yeni çıkmış, genişlemeye başlamış yıldızlar.

    \item \textbf{V — Anakol Yıldızı (Main Sequence / Dwarf)} \\
    Hidrojen füzyonunu çekirdeğinde sürdüren yıldızlar.
    Güneş \textbf{G2V} tipindedir.

    \item \textbf{VI — Süper Cüce (Subdwarf)} \\
    Düşük metalikliğe ve görece düşük parlaklığa sahip sıcak yıldızlar.

    \item \textbf{VII — Beyaz Cüce (White Dwarf)} \\
    Nükleer füzyonun sona erdiği, yoğun ve küçük yıldız kalıntıları.
\end{itemize}

\subsection{Kimyasal Özellik Sınıflandırması}

\begin{itemize}
    \item \textbf{e} : Tayfta \emph{emisyon} çizgileri bulunduğunu gösterir.
    \item \textbf{p} : Yıldızın tayfının \emph{olağandışı (peculiar)} olduğunu gösterir. Kimyasal veya manyetik anomali içerir.
    \item \textbf{n} : Spektral çizgiler \emph{geniş ve bulanık}tır (rotasyon genişlemesi nedeniyle).
    \item \textbf{s} : Spektral çizgiler \emph{keskin ve dar}dır (düşük dönme hızına işaret eder).
\end{itemize}
