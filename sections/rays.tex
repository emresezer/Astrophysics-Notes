\section{Işın Türleri}

Fizikte “ışın” terimi, belirli bir ortamda taşınan enerji akısını ifade eder ve hem elektromanyetik dalgalar hem de parçacık akıları için kullanılır. Evrenin enerji taşınımı çoğunlukla bu iki kategori üzerinden gerçekleşir. Elektromanyetik ışınlar fotonlardan, parçacık ışınları ise maddesel taneciklerden oluşur.

\subsubsection{Elektromanyetik Işın Türleri}

Elektromanyetik ışınlar Maxwell denklemlerinin çözümleridir ve fotonlardan oluşurlar. Genel olarak foton enerjisi
\[
E = h\nu = \frac{hc}{\lambda}
\]
şeklinde ifade edilir.

\paragraph{Semboller:}
\begin{itemize}
    \item $E$: foton enerjisi.
    \item $h$: Planck sabiti.
    \item $\nu$: dalga frekansı.
    \item $c$: ışık hızı.
    \item $\lambda$: dalga boyu.
\end{itemize}

Elektromanyetik ışın türleri aşağıdaki dizilim ile sınıflandırılır:

\begin{itemize}
    \item \textbf{Radyo Dalgaları:} $\lambda \sim 10^{-1} - 10^{6}\ \mathrm{m}$  
    Düşük enerjili, iyonize edici olmayan ışınlardır. Manyetik alan ve plazma fiziğiyle ilişkilidir.

    \item \textbf{Mikrodalgalar:} $\lambda \sim 10^{-3} - 10^{-1}\ \mathrm{m}$  
    Kozmik mikrodalga artalan ışıması bu aralıktadır.

    \item \textbf{Kızılötesi (IR):} $\lambda \sim 10^{-6} - 10^{-3}\ \mathrm{m}$  
    Termal radyasyonun büyük kısmını taşır.

    \item \textbf{Görünür Işık:} $\lambda \sim 380 - 750\ \mathrm{nm}$  
    İnsan gözünün algıladığı bölge.

    \item \textbf{Morötesi (UV):} $\lambda \sim 10^{-9} - 380\ \mathrm{nm}$  
    Enerjik geçişler içerir ve iyonize edici olabilir.

    \item \textbf{X Işınları:} $\lambda \sim 10^{-12} - 10^{-9}\ \mathrm{m}$  
    Yüksek enerjili fotonlar; termal ve yüksek enerji süreçlerinde oluşur.

    \item \textbf{Gama Işınları:} $\lambda < 10^{-12}\ \mathrm{m}$  
    Nükleer süreçler ve relativistik astrofizik olaylarında açığa çıkar.
\end{itemize}

\subsubsection{Parçacık Işın Türleri}

Parçacık ışınları maddesel taneciklerden oluşur ve çoğu durumda relativistik hızlarda hareket ederler. Bir parçacık ışınının kinetik enerjisi
\[
E_k = (\gamma - 1)mc^2
\]
ile verilir.

\paragraph{Semboller:}
\begin{itemize}
    \item $E_k$: parçacığın kinetik enerjisi.
    \item $\gamma$: Lorentz faktörü, $\gamma = \dfrac{1}{\sqrt{1 - v^2/c^2}}$.
    \item $m$: parçacık kütlesi.
    \item $v$: parçacık hızı.
    \item $c$: ışık hızı.
\end{itemize}

Parçacık ışınlarının başlıca türleri:

\begin{itemize}
    \item \textbf{Alfa Işınları ($\alpha$):}  
    İki proton ve iki nötrondan oluşan ağır çekirdek parçacıklarıdır. Düşük penetrasyon fakat yüksek iyonizasyon gücüne sahiptir.

    \item \textbf{Beta Işınları ($\beta^\pm$):}  
    Elektron ($\beta^-$) veya pozitron ($\beta^+$) akılarıdır. Orta seviyede nüfuz edebilirler.

    \item \textbf{Proton Işınları:}  
    Özellikle kozmik ışınların düşük enerjili bileşenlerini oluşturur.

    \item \textbf{Nötron Işınları:}  
    Yük taşımadıkları için madde ile etkileşimleri çekirdeksel süreçler üzerinden gerçekleşir.

    \item \textbf{Kozmik Işınlar:}  
    Protonlar, ağır çekirdekler ve relativistik elektronlardan oluşan yüksek enerjili parçacık ışınlarıdır.

    \item \textbf{Nötrino Işınları:}  
    Etkileşim kesitleri çok küçüktür. Yalnızca zayıf etkileşim üzerinden madde ile etkileşirler.
\end{itemize}

\subsubsection{Işın Türlerinin Karşılaştırılması}

Işın türlerinin karakteristiklerini belirleyen temel büyüklük foton veya parçacık enerjisidir. Elektromanyetik ışınlar için
\[
E = \frac{hc}{\lambda},
\]
parçacık ışınları için ise relativistik ifade
\[
E = \gamma mc^2
\]
geçerlidir.

\paragraph{Semboller:}
\begin{itemize}
    \item $\gamma$: Lorentz faktörü.
    \item $m$: parçacık kütlesi.
    \item $c$: ışık hızı.
    \item $\lambda$: elektromanyetik dalga boyu.
\end{itemize}

\begin{figure}[ht!]
    \centering
    \resizebox{1\textwidth}{!}{\includegraphics{figures/em.png}}
    \caption{Endüktif yük, NASA - kendi kendine oluşturulmuş, NASA tarafından sağlanan bilgiler. NASA'nın EM Spectrum3-new.jpg dosyasından alınmıştır. Kelebek simgesi, P simge setinden alınmıştır. Dosya:P biology.svg İnsanlar, Pioneer plaketinden alınmıştır. Dosya:Human.svg Binalar, Petronas kuleleri ve Empire State binalarıdır. Her ikisi de Skyscrapercompare.svg dosyasından alınmıştır. Milton spektrumunun türünü, dalga boyunu (örneklerle), frekansını ve kara cisim emisyon sıcaklığını gösteren bir diyagramı. Bu dosyanın geliştirilmiş bir sürümüne verilen tepkiyi ölçmek için geçici dosya. NASA'nın bir görüntüsü olan EM Spectrum3-new.jpg dosyasından uyarlanmıştır.
    }
    \label{fig:HRD}
\end{figure}