\section{Dejenere Elektron Basıncı}

Beyaz cüceler gibi düşük sıcaklık ve yüksek yoğunluğa sahip yıldız artıklarında, atomların dış kabuklarındaki elektronlar yoğunluk nedeniyle birbirine çok yaklaşır. Elektronlar \textit{Pauli Dışarlama İlkesi} gereği aynı kuantum durumunu paylaşamazlar. Bu nedenle sistem sıkıştırıldıkça elektronlar daha yüksek momentum seviyelerine geçmek zorunda kalır. Bu durum, sıcaklıktan bağımsız olarak ortaya çıkan ve yıldızın çökmeye karşı koymasını sağlayan bir basınca yol açar. Bu basınca \textbf{dejenere elektron basıncı} denir.

\[
\text{Pauli İlkesi: } \psi_i \neq \psi_j \quad \text{(Aynı kuantum durumunda iki elektron bulunamaz.)}
\]

Elektronların yoğunluğa bağlı olarak Fermi momentumu artar. Fermi enerjisinin tanımı:

\[
E_F = \frac{p_F^2}{2m_e} \quad \text{(non-relativistik rejim)}
\]

Dejenere elektron basıncı non-relativistik yoğunluk rejiminde yaklaşık olarak:

\[
P_{\text{deg}} \propto \left( \frac{\rho}{\mu_e} \right)^{5/3}
\]

Elektronların relativistik hızlara yaklaştığı yüksek yoğunluk durumunda ise güç bağı değişir:

\[
P_{\text{deg}} \propto \left( \frac{\rho}{\mu_e} \right)^{4/3}
\]

Burada:
\[
\rho : \text{yoğunluk}, \qquad \mu_e : \text{bir elektron başına düşen baryon sayısı}
\]

Beyaz cücelerde kütleçekim içeri doğru çökmeye çalışırken, dejenere elektron basıncı dışarı doğru itme sağlar. Denge koşulu:

\[
P_{\text{deg}} = P_{\text{grav}}
\]

Bu denge sürdüğü sürece beyaz cüce stabildir. Ancak kütle arttıkça elektronlar relativistik rejime geçer ve basıncın kütleçekimi dengeleme gücü azalır. Bu durum kritik kütle olan \textbf{Chandrasekhar Limiti} ile sınırlanır:

\[
M_{\text{Ch}} \approx 1.44\,M_\odot
\]

\noindent
\textbf{Sonuç:} Dejenere elektron basıncı, sıcaklıktan bağımsız bir kuantum basıncıdır ve beyaz cücelerin kütleçekim çökmesine karşı koyan temel mekanizmadır. Bu basınç relativistik rejimde zayıfladığında yıldız çöker. Çöküş sırasında elektronlar protonlarla birleşerek nötron oluşturur:

\[
e^- + p^+ \rightarrow n + \nu_e
\]

Bu reaksiyon sonucunda çekirdekte yoğun nötronik madde ortaya çıkar ve yapı \textbf{nötron yıldızının} temelini oluşturur. Çökme devam ederken bu kez yıldızın iç kısmında, yüksek yoğunluk rejiminde etkili olan \textbf{nötron dejenere basıncı} ortaya çıkar. Bu basınç yerçekimine karşı koyarak çöküşü durdurur ve sistemi dengeye getirir.

Sonuç olarak, kütlesi Chandrasekhar limitinin üzerinde olan beyaz cüceler:

\[
\text{Beyaz Cüce} \longrightarrow \text{Nötron Yıldızı}
\]

dönüşümü geçirirler. Ancak bu denge yalnızca belirli bir üst kütle sınırına kadar sürdürülebilir. Nötron yıldızlarının denge sınırı \textbf{Tolman--Oppenheimer--Volkoff (TOV) Limiti} ile belirlenir ve yaklaşık olarak:

\[
M_{\text{TOV}} \approx 2.1 - 2.3\,M_\odot
\]

şeklindedir. Bu sınırın üzerindeki kütlelerde nötron dejenere basıncı da yerçekimini dengeleyemez ve çökme \textbf{kara deliğe} doğru devam eder.
