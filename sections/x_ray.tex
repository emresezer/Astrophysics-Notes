\section{X Işınları}

X ışınları, elektromanyetik tayfın yüksek enerjili ve kısa dalga boylu bölgesine karşılık gelen fotonlardır. Tipik olarak dalga boyları
\[
\lambda \sim 10^{-12} - 10^{-9}\ \mathrm{m},
\]
enerjileri ise
\[
E = h\nu \sim 0.1 - 100\ \mathrm{keV}
\]
aralığındadır. Bu nedenle iyonize edici elektromanyetik radyasyon sınıfında yer alırlar.

\subsubsection*{Fiziksel Tanım}

Elektromanyetik bir dalga olarak X ışını, Maxwell denklemlerinin çözümüdür ve foton enerjisi Planck bağıntısıyla verilir:
\[
E = h\nu = \frac{hc}{\lambda}.
\]

\paragraph{Semboller:}
\begin{itemize}
    \item $E$: foton enerjisi (joule veya elektronvolt).
    \item $h$: Planck sabiti, $6.626\times10^{-34}\ \mathrm{J\,s}$.
    \item $\nu$: frekans (Hz).
    \item $c$: ışık hızı, $3.00\times10^8\ \mathrm{m/s}$.
    \item $\lambda$: dalga boyu (metre).
\end{itemize}

\subsubsection*{Üretim Mekanizmaları}

X ışınlarının oluşum süreçleri yüksek enerji fiziği ve astrofizik açısından temel öneme sahiptir. Başlıca mekanizmalar:

\begin{itemize}
    \item \textbf{Bremsstrahlung (Frenleme Işıması):}  
    Hızlı elektronların iyonlar tarafından Coulomb alanında ivmelendirilmesi sonucu sürekli X-ışını tayfı oluşur.  
    Radyasyon gücü ortalama olarak
    \[
    P \propto Z^2 n_i n_e T^{1/2}
    \]
    ile ölçeklenir.
    
    \item \textbf{Çizgi Spektrumu (Atomik Geçişler):}  
    İç kabuk elektronlarının uyarılması ve tekrar düşük enerji düzeyine inmesi ile karakteristik X ışını çizgileri (örneğin K$_\alpha$, L$_\alpha$) ortaya çıkar.

    \item \textbf{Senkrotron ve Ters Compton Süreçleri:}  
    Relativistik elektronların manyetik alanlarda ivmelenmesi (senkrotron) veya düşük enerjili fotonların relativistik elektronlardan enerji kazanması (IC) X ışını tayfı üretebilir.
\end{itemize}

\subsubsection*{Astrofiziksel Kaynaklar}

X ışınları çoğunlukla yüksek enerji astrofizik ortamlarında ortaya çıkar:

\begin{itemize}
    \item Akresyon diskleri (beyaz cüce, nötron yıldızı, kara delik etrafında)
    \item Sıcak yıldız atmosferleri
    \item Galaksi kümelerindeki milyon kelvinlik gaz
    \item Süpernova kalıntıları
    \item Manyeto-dinamik ortamlar (pulsarlar, magnetarlar)
\end{itemize}

Bu ortamlarda sıcaklık tipik olarak şu aralıktadır:
\[
kT \sim 0.1 - 10\ \mathrm{keV},
\]
dolayısıyla termal X ışını emisyonu baskındır.

\subsubsection*{Madde ile Etkileşim}

X ışınlarının maddenin içinden geçişi üç ana süreçle belirlenir:

\begin{itemize}
    \item \textbf{Fotoelektrik soğurma:} Düşük enerji X ışınlarında baskındır.
    \item \textbf{Compton saçılması:} Orta enerji aralığında belirleyicidir.
    \item \textbf{Çift oluşumu:} $E > 1.022\ \mathrm{MeV}$ durumunda aktif hale gelir.
\end{itemize}

Bu süreçlerin toplam soğurma katsayısı
\[
\mu(E) = \mu_{\mathrm{ph}} + \mu_{\mathrm{C}} + \mu_{\mathrm{pair}}
\]
şeklinde ifade edilir.

\subsection{X Işını Kaynakları}

X ışınları, yüksek enerji fiziksel süreçlerin ürünü olan elektromanyetik fotonlardır. Astrofizik bağlamında X ışını kaynakları, genellikle yüksek sıcaklık, güçlü manyetik alan veya hızlı parçacık ivmelenmesinin bulunduğu ortamlarla ilişkilidir. Kaynaklar termal ve non-termal süreçlere göre sınıflandırılabilir.

\subsubsection{Termal X Işını Kaynakları}

Termal X ışınları, sıcak plazmanın sürekli radyasyonundan ortaya çıkar. Plazma sıcaklığı $T$ ile enerji dağılımı Planck veya Maxwell-Boltzmann istatistiğine bağlıdır. Ortalama foton enerjisi yaklaşık olarak
\[
E \sim kT
\]
ile verilir.

\paragraph{Semboller:}
\begin{itemize}
    \item $E$: foton enerjisi.
    \item $k$: Boltzmann sabiti, $1.38 \times 10^{-23}\ \mathrm{J/K}$.
    \item $T$: plazma sıcaklığı (Kelvin).
\end{itemize}

Başlıca termal X ışını kaynakları:

\begin{itemize}
    \item \textbf{Sıcak Yıldız Atmosferleri:} O ve B tipi yıldızların milyon Kelvin sıcaklıklarındaki korona ve üst atmosferleri X ışını yayar.
    \item \textbf{Galaksi Kümelerindeki Sıcak Gaz:} Kümeler arası milyon Kelvin sıcaklıktaki iyonize gaz, sürekli X ışını yayar.
    \item \textbf{Süpernova Kalıntıları (SNR):} Şoklanmış gazın yüksek sıcaklığı termal X ışını üretir.
\end{itemize}

\subsubsection{Non-Termal X Işını Kaynakları}

Non-termal X ışınları, parçacık ivmelenmesi veya kuantum süreçlerinden ortaya çıkar. Enerji dağılımları genellikle güç yasasına uyar.

\begin{itemize}
    \item \textbf{Senkrotron Işınımı:}  
    Relativistik elektronlar manyetik alan içinde ivmelenir ve X ışını üretebilir. Güç spektrumu:
    \[
    P_\nu \propto N_e B^{(p+1)/2} \nu^{-(p-1)/2}
    \]
    \paragraph{Semboller:}
    \begin{itemize}
        \item $P_\nu$: frekansa bağlı yayılan güç.
        \item $N_e$: elektron yoğunluğu.
        \item $B$: manyetik alan şiddeti.
        \item $p$: elektron enerji spektrumu indeksi.
        \item $\nu$: foton frekansı.
    \end{itemize}

    \item \textbf{Ters Compton (Inverse Compton) Süreci:}  
    Düşük enerjili fotonlar, relativistik elektronlarla çarpışarak yüksek enerjiye ulaşır:
    \[
    E_\gamma \sim \gamma^2 E_{\mathrm{ph}}
    \]
    \paragraph{Semboller:}
    \begin{itemize}
        \item $E_\gamma$: yükseltilmiş foton enerjisi.
        \item $\gamma$: elektron Lorentz faktörü.
        \item $E_{\mathrm{ph}}$: düşük enerjili foton enerjisi.
    \end{itemize}
\end{itemize}

\subsubsection{Akresyon Diskleri ve Kompakt Nesneler}

X ışınlarının en güçlü astrofizik kaynakları, kompakt nesneler etrafındaki akresyon diskleridir:

\begin{itemize}
    \item \textbf{Beyaz Cüce Akresyon Diskleri:} Termal ve bazen non-termal X ışını üretir.
    \item \textbf{Nötron Yıldız ve Pulsar Sistemleri:} Yüzey sıcaklığı ve manyetik alan ivmelenmeleri X ışını yayar.
    \item \textbf{Kara Delik Akresyon Diskleri:} X ışını tayfı disk sıcaklığına ve relativistik etkilerle Doppler kaymasına bağlıdır.
\end{itemize}

\subsubsection{Diğer Kaynaklar}

\begin{itemize}
    \item Magnetarlar ve yüksek manyetik alanlı pulsarlar.
    \item Kısa süreli patlamalar: Gamma-ray burst sonrası X ışını “afterglow”.
    \item Termal olmayan galaksi çekirdekleri (Active Galactic Nuclei, AGN) jet ve disk yapıları.
\end{itemize}

\subsection{X Işını Astrofiziğinde Foton Yayınımı}

X ışını astrofiziğinde fotonların yayınımı, kaynakların fiziksel koşulları ve enerji dağılımlarına bağlıdır. X ışını fotonları hem termal hem de non-termal süreçlerden gelir ve spektral dağılımları farklı karakteristikler taşır.

\subsubsection{Termal Yayınım (Bremsstrahlung ve Siyah Cisim)}

Sıcak plazma, elektronların iyonlarla etkileşimi sonucu sürekli X ışını yayar (frenleme ışınımı, Bremsstrahlung). Yüksek sıcaklıktaki plazmanın tayf yoğunluğu yaklaşık olarak
\[
\epsilon_\nu \propto n_e n_i Z^2 T^{-1/2} e^{-h\nu/kT}
\]
ile verilir.

\paragraph{Semboller:}
\begin{itemize}
    \item $\epsilon_\nu$: birim hacimde frekans başına yayılan enerji yoğunluğu.
    \item $n_e$: elektron yoğunluğu.
    \item $n_i$: iyon yoğunluğu.
    \item $Z$: iyon yük sayısı.
    \item $T$: plazma sıcaklığı (K).
    \item $\nu$: foton frekansı.
    \item $h$: Planck sabiti.
    \item k: Boltzmann sabiti.
\end{itemize}

Termal plazmada enerji dağılımı Maxwell-Boltzmann istatistiğine uygundur. Ortalama foton enerjisi:
\[
\langle E_\gamma \rangle \sim kT
\]

\subsubsection{Çizgi Yayınım (Atomik ve Nükleer Geçişler)}

İç kabuk elektronlarının uyarılması ve tekrar düşük enerji düzeyine dönmesi X ışını çizgileri üretir. Karakteristik X ışını çizgileri genellikle K ve L serisi olarak adlandırılır:
\[
E_{K\alpha} = E_{2p} - E_{1s}
\]

\paragraph{Semboller:}
\begin{itemize}
    \item $E_{K\alpha}$: K$_\alpha$ foton enerjisi.
    \item $E_{2p}$: ikinci kabuk enerji seviyesi.
    \item $E_{1s}$: birinci kabuk enerji seviyesi.
\end{itemize}

\subsubsection{Non-Termal Yayınım (Senkrotron ve Inverse Compton)}

Relativistik elektronlar manyetik alan içinde hareket ederken senkrotron X ışını yayabilir. Güç spektrumu:
\[
P_\nu \propto N_e B^{(p+1)/2} \nu^{-(p-1)/2}
\]

\paragraph{Semboller:}
\begin{itemize}
    \item $P_\nu$: frekansa bağlı yayılan enerji.
    \item $N_e$: elektron yoğunluğu.
    \item $B$: manyetik alan şiddeti.
    \item $p$: enerji spektrumu indeksi.
    \item $\nu$: foton frekansı.
\end{itemize}

Düşük enerjili fotonların relativistik elektronlarla çarpışması sonucu ters Compton (Inverse Compton) yayınımı oluşur:
\[
E_\gamma \sim \gamma^2 E_{\mathrm{ph}}
\]

\paragraph{Semboller:}
\begin{itemize}
    \item $E_\gamma$: yükseltilmiş foton enerjisi.
    \item $\gamma$: elektron Lorentz faktörü.
    \item $E_{\mathrm{ph}}$: düşük enerjili foton enerjisi.
\end{itemize}

\subsubsection{Akresyon Disklerinden X Işını Yayınımı}

Kompakt nesnelerin (kara delik, nötron yıldızı) etrafındaki akresyon disklerinde sıcaklık $T(r)$, disk yarıçapına bağlı olarak değişir. Yayınım yüzey yoğunluğu:
\[
F(r) = \frac{3GM\dot{M}}{8\pi r^3} \left[1 - \left(\frac{R_\ast}{r}\right)^{1/2}\right]
\]

\paragraph{Semboller:}
\begin{itemize}
    \item $F(r)$: yüzey birimi başına enerji akısı.
    \item $G$: evrensel yerçekimi sabiti.
    \item $M$: kompakt nesne kütlesi.
    \item $\dot{M}$: kütle akış hızı.
    \item $r$: disk yarıçapı.
    \item $R_\ast$: kompakt nesne yarıçapı (nötron yıldızı veya kara delik iç sınır).
\end{itemize}

Bu dağılım, disk boyunca sıcaklığın değişimi ile farklı X ışını enerjilerini üretir. Diskin iç bölgeleri yüksek enerjili X ışınları üretirken dış bölgeler düşük enerjili fotonlar yayar.
