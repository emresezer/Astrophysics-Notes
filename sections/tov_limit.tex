\section{Tolman-Oppenheimer-Volkoff (TOV) Limiti}

Tolman-Oppenheimer-Volkoff (TOV) limiti, bir nötron yıldızının kendi kütleçekimsel çökmesine karşı dayanabileceği \textbf{en büyük kütleyi} tanımlar. Bu limit, bir yıldızın çekirdeğinin artık nükleer füzyon yapamadığı ve tamamen \textit{nötron dejenere basıncına} bağlı olarak ayakta kaldığı durumlarda geçerlidir. 

Einstein'ın Genel Görelilik teorisine göre, kütleçekim kuvveti yalnızca kütleye değil aynı zamanda \textbf{enerji yoğunluğu ve basınca} da bağlıdır. Bu nedenle nötron yıldızlarının denge koşulu, klasik hidrostatik denge denklemi yerine \textbf{TOV denklemi} ile ifade edilir.

\subsection{TOV Denklemi}

Bir nötron yıldızının denge koşulu şu şekilde verilir:
\begin{equation}
\frac{dP(r)}{dr} = - \frac{G}{r^2} \left( \rho(r) + \frac{P(r)}{c^2} \right) \left( M(r) + 4\pi r^3 \frac{P(r)}{c^2} \right) \left( 1 - \frac{2GM(r)}{rc^2} \right)^{-1},
\end{equation}

burada:
\begin{itemize}
    \item $P(r)$ — yarıçap $r$'deki basınç,
    \item $\rho(r)$ — kütle yoğunluğu,
    \item $M(r)$ — yarıçap $r$ içindeki toplam kütle,
    \item $G$ — evrensel kütleçekim sabiti,
    \item $c$ — ışık hızı.
\end{itemize}

Bu denklem, kütleçekimi ve dejenere nötron basıncının nasıl denge kurduğunu ifade eder.

\subsection{TOV Kütle Limiti}

Nötron dejenere basıncı, nötronların Pauli Dışarlama İlkesi nedeniyle aynı kuantum durumunu paylaşamamasından kaynaklanan bir \textit{kuantum basıncıdır}. Ancak kütle arttıkça kütleçekim kuvveti daha baskın hale gelir ve bir noktadan sonra bu basınç denge sağlayamaz.

Bu kritik kütle:
\begin{equation}
M_{\text{TOV}} \approx 2.0 - 3.0 \, M_{\odot}
\end{equation}

aralığındadır (güncel nötron yıldızı denklemleri ve nükleer maddenin durumu belirsiz olduğundan tam değer kesin değildir).

\subsection{TOV Limitinin Aşılması}

\begin{itemize}
    \item Eğer $M < M_{\text{TOV}}$ ise: Nötron yıldızı kararlı kalır.
    \item Eğer $M > M_{\text{TOV}}$ ise: Nötron dejenere basıncı artık kütleçekime karşı koyamaz ve yıldız \textbf{kütleçekimsel olarak çöker}.
\end{itemize}

Dolayısıyla TOV limiti, doğrudan bir yıldız kalıntısının kara delik olup olmayacağını belirleyen kritik eşiktir.