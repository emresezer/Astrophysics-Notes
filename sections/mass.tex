\section{Yıldızlarda Kütle, Işınım ve Ömür İlişkisi}

Bir yıldızın temel özellikleri büyük ölçüde kütlesi tarafından belirlenir. Kütle arttıkça çekim kuvveti ve çekirdekteki basınç artar. Bu durum nükleer füzyon hızını yükseltir ve yıldızın toplam ışınım gücünü (aydınlatma gücü) belirgin biçimde büyütür. Yıldızlardaki ışınım gücü ile kütle arasındaki yaklaşık ilişki:

\[
L \propto M^{3.5}
\]

Burada:
\[
L : \text{yıldızın ışınım gücü (ışıma şiddeti)}, \qquad M : \text{yıldızın kütlesi}
\]

Bu ifade, yıldızın kütlesi iki katına çıktığında ışınım gücünün yaklaşık \(2^{3.5} \approx 11\) kat arttığını gösterir. Dolayısıyla daha kütleli yıldızlar çok daha parlak ve iç reaksiyonları çok daha hızlıdır.

Yıldızın ana kol (main-sequence) ömrü, toplam yakıt miktarının kütle ile, yakıt tüketim hızının ise ışınım gücü ile orantılı olması nedeniyle yaklaşık olarak:

\[
\tau \propto \frac{M}{L}
\]

Yukarıdaki kütle-ışınım ilişkisi kullanıldığında:

\[
\tau \propto \frac{M}{M^{3.5}} = M^{-2.5}
\]

Sonuç olarak kütle arttıkça ömür kısalır.
Daha kütleli yıldızlar daha parlak olmalarına rağmen yakıtlarını çok hızlı tüketirler. Buna karşılık, düşük kütleli yıldızlar füzyonu çok yavaş sürdürürler ve evrende trilyonlarca yıl yaşayabilirler.

Özetle:
\[
L \propto M^{3.5}, \qquad \tau \propto M^{-2.5}
\]

Yüksek kütle $\rightarrow$ yüksek ışınım $\rightarrow$ hızlı yakıt tüketimi $\rightarrow$ kısa ömür.

Düşük kütle $\rightarrow$ düşük ışınım $\rightarrow$ yavaş yakıt tüketimi $\rightarrow$ uzun ömür.
