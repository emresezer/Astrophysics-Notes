\section{Beyaz Cüce}

Güneşe göre düşük ve orta kütleli yıldızlar ($0.08 - 8\,M_\odot$) oluşturur. Kütlesi $0.5 - 1.4\,M_\odot$ aralığındadır. Yarıçapı $\approx 0.01\,R_\odot$ mertebesindedir. Yoğunluğu $10^6 - 10^9\,\text{g/cm}^3$ aralığındadır. Yüzey sıcaklığı doğduğunda $\approx 10^5\,\text{K}$ olur ve zamanla soğur.

Reaksiyon yapmaz ve enerji üretmez. Çekirdeğinde çoğunlukla C ve O bulunur. Daha düşük kütlelerde He, daha yüksek kütlelerde O, Ne ve Mg bulunabilir. Elektron dejenere basıncı kütleçekimsel basınca karşı koyar.

$10^9 - 10^{10}$ yıl sonra Siyah Cüceye dönüşürler. Siyah Cüce şu anda teoriktir çünkü bir Siyah Cüce oluşması için evrenin yaşı yetersizdir.

 
\subsection{Beyaz Cüce'nin Soğuma Süreci}
 Beyaz Cüce enerji üretemez. Ana kol evresindeki enerjisini zamanla tüketir. Isı kapasitesi, yoğun ve küçük hacimli olduğu için düşüktür. Isı ve ışık yayarak(radyasyon) enerji kaybeder. Stefen-Boltzman yasaına uygun davranır.
