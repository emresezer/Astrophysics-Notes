\section{Yıldız}

 Ağırlıklı olarak H ve He'dan oluşan, ışık ve ısı yayan, kararlı plazmalardır. Çekirdeklerinde füzyon meydana gelir. Açığa çıkan
enerji yıldızın yüzeyine ulaşır ve dış uzaya radyasyon ile yayılır. Yıldızın tayfı(spektrumu) ve parlaklığı ile kütlesi, yaşı, kimyasal bileşeni vb. özellikleri belirlenebilir. Birim olarak güneş kütlesi, güneş yarıçapı ve güneş parlaklığı kullanılır.
