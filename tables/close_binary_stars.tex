\begin{table}[h!]
\centering
\begin{tabular}{|p{0.22\textwidth}|p{0.22\textwidth}|p{0.50\textwidth}|}
\hline
\textbf{1. Bileşen} & \textbf{2. Bileşen} & \textbf{Sistem Sonucu} \\
\hline
Kırmızı Dev & Ana Kol Yıldızı & Geniş çift sistem; kütle aktarımı yok. \\
\hline
Kırmızı Dev & Ana Kol Yıldızı & Yakın çift sistem; Roche taşması ile kütle transferi. \\
\hline
Beyaz Cüce & Ana Kol Yıldızı & Akresyon → Nova; $M_{\mathrm{Ch}}$ aşılırsa Tip Ia Süpernova. \\
\hline
Nötron Yıldızı & Ana Kol Yıldızı & X-ışını ikilisi; akresyon diski yüksek enerjili yayım üretir. \\
\hline
Nötron Yıldızı & Nötron Yıldızı & Birleşme → Gravitasyonel dalga + Kilonova. \\
\hline
Kara Delik & Ana Kol Yıldızı & Mikrokuasar; jetler ve güçlü akresyon diski. \\
\hline
Kara Delik & Kara Delik & Gravitasyonel dalga ile birleşme olayı (GW sinyali). \\
\hline
\end{tabular}
\caption{Çift yıldız bileşenlerine göre evrimsel sonuçlar.}
\end{table}
